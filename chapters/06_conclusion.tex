% !TeX root = ../main.tex
% Add the above to each chapter to make compiling the PDF easier in some editors.

\chapter{Conclusion}\label{chapter:conclusion}

In this master thesis, we presented the Benchy Viewer, a web-based serverless application that offers a dynamic approach to exploring and understanding performance data. 

The primary goal of this project was to empower database engineers with a tool that facilitates in-depth analysis of query execution. The Benchy Viewer successfully realizes this objective by providing an interactive and intuitive platform for exploring performance benchmark data.

The Benchy Viewer introduces a new level of interactivity to the analysis process. Unlike its predecessor, Benchy, which presented static visualizations, the Benchy Viewer incorporates more visualizations with interactive functionalities. The drag-and-drop feature in the Analytics Dashboard further enhances user experience, enabling a flexible and personalized exploration of analytical configurations.

The Benchy Viewer offers a diverse range of analytical perspectives, providing users with multiple ways to inspect and interpret benchmark results. The Analytics Dashboard incorporates various chart types, including bar charts, violin plots, tables, and scatter plots, each offering unique insights. The Query Plan View and Input File View contribute additional perspectives, allowing users to delve into query plans and explore raw benchmark data in a tabular format.

The ability to save and load analytical configurations enhances collaboration among database engineers. By providing a user-friendly mechanism to download and upload configuration files, the Benchy Viewer enables seamless sharing of insights and configurations. The adaptable interface and preconfigured analytical contexts make it convenient for one user to share meaningful setups with others, fostering efficient collaborative efforts.

The Benchy Viewer lays the groundwork for future research and development in several areas. The integration of components from the Query Plan Difference Visualizer and the Plotly charting library posed complexities, necessitating careful consideration and adjustments. Some features, such as synchronized zooming and sorting functionalities, are currently absent and could be areas for future development. Integration of additional features from the Query Plan Difference Visualizer, such as plan similarity metrics, could enhance the tool's capabilities. The exploration of global zoom functionality and sorting options in specific scenarios represents avenues for further improvement in the Analytics Dashboard.

One notable strength of the Benchy Viewer is its flexibility and adaptability. The system design, following the model-view-viewmodel paradigm, allows for the seamless integration of various charting libraries. This flexibility ensures that the Benchy Viewer can evolve by incorporating new visualization resources, addressing potential limitations of the current charting library Plotly, and expanding its spectrum of supported visualizations.

In conclusion, the Benchy Viewer represents a valuable contribution to the field of database performance analysis. Its interactive features, diverse analytical perspectives, and collaboration support make it a powerful tool for database engineers seeking to optimize query execution. While there are challenges and areas for future improvement, the Benchy Viewer's strengths position it as a foundation for ongoing research and development, with the potential to further enhance its analytical capabilities and adapt to evolving needs in the realm of database performance profiling.


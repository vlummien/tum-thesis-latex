% !TeX root = ../main.tex
% Add the above to each chapter to make compiling the PDF easier in some editors.

\chapter{Discussion}\label{chapter:discussion}

This thesis aims to support database engineers in optimizing query execution by providing an interactive user interface, enabling an effective in-depth analysis process. The achieved objectives include the development of an interactive platform enabling users to inspect results through diverse metrics and charts. 

\section{Evaluation of Achievement of Objectives}

The analysis process should lead to the identification of performance bottlenecks.
This theses does not guarantee the identification of bottlenecks, however it helps database engineers to find them through various visualizations and perspectives within the Benchy Viewer.


\subsection{Interactive Analysis}
% Explain
The earlier performance measurements visualization method Benchy \parencite*{benchy},  presents a range of visualizations for benchmark data, incorporating diverse metrics and the capability to illustrate different instances of database systems.\\
However, this approach prompts the need for dynamic interactions with the resulting data. For instance, in violin plots, each query performance result is portrayed as a data point within a violin, lacking information about which query a data point corresponds to. Due to the static nature of visualizations in the preceding approach, the exact values of query performance results remain concealed. In contrast, the Benchy Viewer introduces a dynamic element, allowing users to hover over data points to reveal precise details, such as time in milliseconds or the speedup/slowdown factor.


There are some common practices in statistical dashboards that could enhance the Benchy Viewer's utility, but are currently not available.\\
For instance, when multiple charts of the same type display specific query results with different metrics (e.g., three bar charts showcasing execution, compilation, and total time in milliseconds), zooming into one chart should automatically trigger a synchronized zoom in the others. Presently, users have to manually apply the same zoom to each chart individually, as this functionality is not supported by the Benchy Viewer.\\
Sorting features are also absent in the Benchy Viewer. In a bar chart displaying individual query results, users cannot sort data in ascending or descending order. Additionally, within the Input File Reader the benchmark data comprises for some metrics an array of multiple values (e.g., an array of total time results contributing to the average total time for a query), and the Benchy Viewer does not provide a sorting functionality within those arrays.\\
The only sorting feature available in the Benchy Viewer is offered through the slowdown and speedup tables introduced in Section \ref{sec:tables}. 
From the baseline system's perspective, these tables sort the maximum slowdown in ascending order and the maximum speedup in descending order.

Despite the Benchy Viewer lacking certain analytical features, it still offers a valuable set of interactive tools.
Therefore, our Analytics Dashboard excels beyond the preceding visualization approach provided by Benchy. Through its core user interface design, including the sidebar and header, the Benchy Viewer offers interactive features in a clear and intuitive manner. It facilitates working with data interactively, enhancing the analysis processes for database developers, and aiding in the identification of  performance bottlenecks with all its assisting tools. 

In addition, the Benchy Viewer boasts further interactive features, allowing users to fine-tune data visualizations to suit their preferences, facilitating the examination of performance measurements from diverse angles.


\subsection{Multiple Perspectives}

The Benchy Viewer presents a flexible approach of analyzing benchmark results, offering diverse perspectives through a range of charts and plots. Users can create more perspectives with these visualizations by selecting specific metrics and adjusting chart scales, including linear, logarithmic, and throughput options.
Moreover, the Query Plan View plays a pivotal role in inspecting results from another angle, offering new insights compared to what the Analytics Dashboard provides with its charts and plots. To complete the spectrum of perspectives, the Input File View enables users to inspect data in a tabular and straightforward format.

But the landscape of perspectives within the Benchy Viewer is not complete. Additional visualizations and metrics could enhance its capabilities further. For instance, the scale perspective is currently not covered by the Benchy Viewer. At present, there are no visualizations for comparing data based on the scale factor, suggesting a potential area for future development.\\
Additionally, the Query Plan Difference Visualizer \parencite*{semantic-diff} has not been fully integrated into the Benchy Viewer, lacking features such as the ability to display query plan similarity.

Nevertheless, the Benchy Viewer currently offers a diverse range of analytical perspectives. This multiplicity of viewpoints contributes to a more comprehensive understanding and interpretation of the results. The combination of these features significantly enhances the Benchy Viewer's analytical capabilities, aligning with our objective of identifying database performance bottlenecks.



% - Mehr Perspektiven wären möglich: Andere Visualisierungen, mehr Metriken

% - Mit Query Plan Difference Visualizer vergleichen: (Wir haben nicht alle Funktionen, kann man aber in Zukunft implementieren, Z.B. Hat das andere Tool eine universelle Bar Chart Visualization, um alle Metriken anzuzeigen, die es in den Daten gibt -> Benchy Viewer kann nur ein paar anzeigen -> Vllt das in Future Section)

\subsection{Collaboration}

The ability to save and load analytical configurations stands as a fundamental element in fostering collaboration. Instead of using a database, the Benchy Viewer achieves the saving feature by allowing users to download a file containing the application state. This file can then be easily uploaded from any machine running the Benchy Viewer.\\
Moreover, the adaptable interface hub enables the creation of meaningful analytical contexts, making it particularly conducive to collaboration. In a collaborative scenario, one individual can uncover specific insights and swiftly share them with others by providing a preconfigured and meaningful setup. 

On the one hand, there is a preference to save configurations online on the respective platform. On the other hand, some might opt to save it as a file on their local machine, benefiting from the simplicity of avoiding account creation and login complexities.

However, providing at least one of these approaches is enough to enable this function to support collaboration. Hence, the Benchy Viewer achieves this need through a user-friendly approach that contributes to the efficiency of collaborative efforts, ensuring a seamless and uncomplicated process for sharing information among users.
  

\section{Challenges and Technical Limitations}

\subsubsection*{Integration of the Query Plan Difference Visualizer}
% Diff
Integrating components from the Query Plan Difference Visualizer that utilized a different state management tool posed a significant challenge. The original app's state logic, actions, and data flow were intricately tied to its specific state management paradigm. Adapting these components to work seamlessly with the state management tool employed in the Benchy Viewer required careful consideration and adjustments to ensure a harmonious integration.\\
Achieving consistency in the flow of data between components and the state was a paramount concern during the integration process. The disparities in how data was managed and updated led to conflicts and inconsistencies. Ensuring a reliable and predictable data flow across the integrated components demanded thorough testing and iterative refinement to identify and resolve any unforeseen issues.

\subsubsection*{Integration of the Charting Library}

Integrating Plotly, a powerful charting library, into the Benchy Viewer presented a multifaceted challenge. Plotly operates on its own set of visualization principles, and ensuring compatibility and smooth integration with existing components required thoughtful examination of data structures, styling, and interactive features.\\
As the complexity and volume of data increased, performance considerations became a crucial factor. Ensuring smooth rendering and responsiveness of Plotly charts, especially when dealing with large datasets, posed technical challenges that needed careful optimization and testing. 

% - Drag and Drop: Is complex to develop, bc of topics, such as collision detection, Layout challenges (Tiles have to be the same size), state challenges (complex state, has to be connected to business logic).


\subsubsection*{Integration of Drag-and-Drop}
Maintaining a consistent and intuitive user experience with the drag-and-drop feature in the Analytical Dashboard was a central challenge. Integrating dnd kit \parencite*{dndkit} seamlessly into the UI and the state management of the Benchy Viewer demanded careful consideration of user expectations and behaviors. For instance, circumventing layout limitations regarding fixed and same-sized drag-and-drop elements posed a specific challenge. To ensure behavioral demands, employing aspects such as collision detection and dynamic styling when dragging elements were necessary. These deliberations enabled the overall usability of the drag-and-drop feature, making it an integral and user-friendly part of the Benchy Viewer.


\section{Potentials and Future Improvements}

The Benchy Viewer, while achieving its objectives in providing an interactive and insightful platform for performance benchmark analysis, also acknowledges challenges and considers avenues for future growth. Through ongoing development and a commitment to address the amount of functionalities and technical limitations, the Benchy Viewer is poised for continued effectiveness and relevance in the landscape of database performance profiling.

The Benchy Viewer was developed as part of a prototype, providing fundamental core functionality that lays the groundwork for future development. As it stands, the integration of the Query Plan Difference Visualizer opens the door for further enhancements by incorporating additional features. One notable feature that remains to be integrated is the display of a plan similarity metric, offering the potential for valuable additional insights into the comparative analysis of query plans. This future development could significantly augment the capabilities of the Benchy Viewer, enriching its functionality and utility in query plan analysis.

Additional improvement potentials lie within the Analysis Dashboard. The enhancement of Plots and Charts could involve the integration of new features, such as global zoom functionality. This feature would enable synchronized zooming across multiple charts, providing analysts with the capability to effortlessly examine different facets of the data in greater detail.\\
Moreover, introducing sorting functionalities in specific scenarios could prove advantageous. For instance, having the ability to sort results based on certain criteria would be beneficial for analyzing queries with varying execution times. Database developers could initiate their analysis by examining the longest-running queries and then progressively move on to the next in line, streamlining the investigative process.

In the future, when custom visualizations are envisioned that may go beyond the capabilities of Plotly, the Benchy Viewer can seamlessly integrate other charting libraries to broaden its visualization spectrum. This flexibility allows for the incorporation of diverse visualization tools when Plotly presents limitations.\\
The system design of the Benchy Viewer adheres to the model-view-controller paradigm, strategically segregating the user interface from the underlying business logic. This architectural choice not only enhances maintainability and modularity but also facilitates the seamless integration of various charting libraries. As a result, the Benchy Viewer remains adaptable and open to incorporating additional visualization resources in pursuit of richer and more diverse data representations.







